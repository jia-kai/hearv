% $File: intro.tex
% $Date: Sat Nov 29 11:01:36 2014 +0800
% $Author: jiakai <jia.kai66@gmail.com>

\section{引言}
声音的本质是震动,人的听觉也是由于空气带动鼓膜震动进而引发一系列神经反应
而形成的。声波在媒介中传播,当它碰到物体后,会使物体表面产生微小震动。
取决于各种条件,物体表面可能会跟随环绕的媒介一起移动或者因它的震动模式变形。
在这两种情况下,运动模式包含了可用于恢复声音或了解的物体结构的信息。

物体因声音而导致的振动在近年来已被用于远程声音获取,
在监视和安全领域有重要的作用,如远程窃听谈话。
现有的采集声音的方法都是主动的,需将激光束或图案投影到物体的振动表面。

Davis等人指出\cite{Davis2014VisualMic},物体振动往往能产生足够的视觉信号,
通过使用一个高速摄像机就可以部分恢复产生它们的声音。 
这一方法恢复的声音的质量不如主动方法,但是它据有一些优点,比如不需要主动的光照、
不需要物体具有良好的材质等。由于在视频的每一像素观测得到了声音的空间测量,
这一方法还可以用于其他分析,比如声音导致的物体形变等。

而\cite{Davis2014VisualMic}的主要篇幅集中在基于高速摄影的全局运动恢复与
对恢复出的音频信号的后处理,对于利用卷帘快门(rolling shutter)从商用数码相机
的视频中恢复声音的技术仅进行了简单的描述。

在本SRT中,我们尝试在非实验室条件下,使用普通设备与商用数码相机,
从视频中基于物体表面的震动来被动地恢复拍摄视频时的环境声音。

% vim: filetype=tex foldmethod=marker foldmarker=f{{{,f}}}

